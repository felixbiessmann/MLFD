% !Tex root = main.tex
\newpage
\section{Theory}
\emph{Functional dependencies}(FD) are a way of expressing ``\emph{a priori} knowledge of restrictions or constraints on permissible sets of data''\cite[p.42]{MAI83} in relational database theory. In order to give a definition of FDs, some concepts stemming from relational database theory need to be introduced beforehand.

\subsection{Relational Database Theory}
A \emph{relation scheme} $R$ is a finite set of \emph{attribute names} ${A_1, A_2, \dots, A_n}$, with each attribute name $A_i$ having a corresponding set $D_i$, $1 \leq i \leq n$, called \emph{domain} of $A_i$. Let $\boldsymbol{D} = D_1 \cup D_2 \cup \dots \cup D_n$, then a \emph{relation} $r$ on relation scheme $R$ is a finite set of mappings ${t_1, t_2, \dots, t_p}$ from $R$ to $\boldsymbol{D}$. We call those mappings \emph{tuples} under the restriction that $t(A_i) \subseteq D_i$, $1 \leq i \leq n$.\cite[p.2]{MAI83}


\subsection{Definition of a Functional Dependency}
Therefore, a relation $r$ on scheme $R$ with subsets $X$ and $Y$ of $R$ is considered. The relation $r$ is said to 
\subsection{Properties of Functional Dependencies}

