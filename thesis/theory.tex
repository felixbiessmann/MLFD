% !Tex root = main.tex
\newpage
\section{Theory}
\emph{Functional dependencies}(FDs) are a way of expressing ``a priori knowledge of restrictions or constraints on permissible sets of data''\cite[p.42]{MAI83} in relational database theory.
In order to give a definition of FDs, some concepts stemming from relational database theory need to be introduced beforehand.

\subsection{Relational Database Theory}
A \emph{relation scheme}\footnote{also called \emph{relational schema} in literature\cite[p.21]{ABE19}} \boldsymbol{R} is a finite set of \emph{attribute names} $\{A_1, A_2, \dots, A_n\}$, where to each attribute name $A_i$ corresponds a set $D_i$, called \emph{domain} of $A_i$, $1 \leq i \leq n$. \\

Let $\boldsymbol{D} = D_1 \cup D_2 \cup \dots \cup D_n$, then a \emph{relation} $r$ on relation scheme \boldsymbol{R} is a finite set of mappings $\{t_1, t_2, \dots, t_p\}$ from \boldsymbol{R} to $\boldsymbol{D}$:
\begin{align}
  &t_i: \boldsymbol{R} \to \boldsymbol{D},
\end{align}
where we call those mappings \emph{tuples} under the constraint that \cite[p.2]{MAI83}
\begin{align}
  t(A_i) \subseteq D_i.
\end{align}

\subsection{Definition of a Functional Dependency}
Consider a relation $r$ on scheme \boldsymbol{R} with subset $X \subseteq \boldsymbol{R}$ and a single attribute $A_i \in \boldsymbol{R}$.
A FD $X \to A$ is said to be \emph{valid} in $r$, if and only if
\begin{align}\label{eq:fd-condition}
  t_i[X] = t_j[X] \Rightarrow t_i[A] = t_j[A]
\end{align}
holds for all all pairs of distinct tuples $t_i,t_j \in r$.\cite[p.~21]{ABE19} 
We say that $X$ \emph{functionally determines} $A$\cite[p.~43]{MAI83} and name $X$ the \emph{left side}, whilst calling $A$ the \emph{right side}.

\subsection{Approximate Functional Dependencies}
In the field of data profiling an extensive body of theory and algorithms for FD detection has been created in the past decades.\cite{PAP15}
These mainly consider FDs as defined in formula \ref{eq:fd-condition}.
Howevever, the strict detection of FDs yields results that are solely applicable in a strictly controlled environment.
Real-world datasets faced by data-scientists or database engineers are often \emph{noisy}.
Entries might be corrupted by missing data, wrongly entered entries or incomplete datasets.
Inconsistencies are to be expected.
Thus, functionally dependent column-combinations might not be detected as such. This may result in misleading insights when searching for FDs. \\

To illustrate this, table \ref{tab:example-afd-necessity} shows an example of noisy data.
The potential FD \textbf{Town} $\to$ \textbf{ZIP} is not captured by the definition given in equation \ref{eq:fd-condition}.
Due to a type-error, the potential FD is invalidated.
To still capture meta-information, a different dependency-measure than given in equation \ref{eq:fd-condition} is needed. \\

\emph{Approximate FDs} (AFDs), sometimes called \emph{Relaxed FDs}, improve the applicability of FDs, ``in that they relax one or more constraints of the canonical FDs''\cite[p.~1]{CAR16}. While there are AFDs introducing general error measures, others are defined ``aiming to solve specific problems''\cite[p.~1]{CAR16}. \\

\begin{table}[h]
	\centering
	\begin{tabular}{lcccc}
		\toprule
		& \multicolumn{3}{c}{Data} \\ \cmidrule(lr){2-5}
        ID & First name & Last name & Town & ZIP \\
		\midrule
        1 & Alice & Smith & Munich & 19139 \\
        2 & Peter & Meyer & Muinch & 19139 \\
        3 & Hannah & Parker & Munich & 19139  \\
        4 & John & Pick & Berlin & 12055 \\
		\bottomrule
	\end{tabular}
    \caption{Even though the ZIP-Code functionally determines the town (and vice-versa) in the given example, a FD is not capable of displaying this fact. A type-error in the dataset with ID 2 invalidates the functional dependency.}
	\label{tab:example-afd-necessity}
\end{table}

The error measure for 
