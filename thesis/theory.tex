% !Tex root = main.tex
\newpage
\section{Theory}
\emph{Functional dependencies}(FD) are a way of expressing ``\emph{a priori} knowledge of restrictions or constraints on permissible sets of data''\cite[p.42]{MAI83} in relational database theory.
In order to give a definition of FDs, some concepts stemming from relational database theory need to be introduced beforehand.

\subsection{Relational Database Theory}
A \emph{relation scheme}\footnote{also called \emph{relational schema} in literature\cite[p.21]{ABE19}} $R$ is a finite set of \emph{attribute names} $\{A_1, A_2, \dots, A_n\}$, where to each attribute name $A_i$ corresponds a set $D_i$, called \emph{domain} of $A_i$, $1 \leq i \leq n$. \\

Let $\boldsymbol{D} = D_1 \cup D_2 \cup \dots \cup D_n$, then a \emph{relation} $r$ on relation scheme $R$ is a finite set of mappings $\{t_1, t_2, \dots, t_p\}$ from $R$ to $\boldsymbol{D}$:
\begin{align}
  &t_i: R \to \boldsymbol{D},
\end{align}
where we call those mappings \emph{tuples} under the constraint that \cite[p.2]{MAI83}
\begin{align}
  t(A_i) \subseteq D_i.
\end{align}

\subsection{Definition of a Functional Dependency}
For giving a definition of a FD, relation $r$ on scheme $R$ with subset $X \subseteq R$ and a single attribute $A_i \in R$ are considered.
A FD $X \to A$ is said to be \emph{valid} in $r$, if and only if
\begin{align}\label{eq:fd-condition}
  t_i[X] = t_j[X] \Rightarrow t_i[A] = t_j[A]
\end{align}
holds for all all pairs of distinct tuples $t_i,t_j \in r$.\cite[p.21]{ABE19} 
We say that $X$ \emph{functionally determines} $A$\cite[p.43]{MAI83} and name $X$ the \emph{left side}, whilst calling $A$ the \emph{right side}.

\subsection{Approximate Functional Dependencies}
In the field of data profiling an extensive body of theory and algorithms for FD detection has been created in the past decades.
These mainly consider FDs as defined in equation \ref{eq:fd-condition}.
Howevever, the strict detection of FDs yields results that are solely applicable in a strictly controlled environment.
Real-world datasets faced by data-scientists or database engineers are often \emph{noisy}.
Entries might be spelled incorrectly and inconsistencies are to be expected. \\

Here goes a nice example explaining table \ref{tab:example-afd-necessity}
\emph{Approximate FDs (AFDs)} relax the strict definition of FDs and introduce an error-measure.  

\begin{table}[h]
	\centering
	\begin{tabular}{lcc}
		\toprule
		& \multicolumn{2}{c}{Data} \\ \cmidrule(lr){2-3}
    First name & Last name & ZIP \\
		\midrule
    Alice & Smith & 19139 \\
		Pencil & 1 & big \\
		Marker & 4 &  \\
    Fountain Pen & 43 & green \\
		\bottomrule
	\end{tabular}
	\caption{AFD example I have to work out.}
	\label{tab:example-afd-necessity}
\end{table}

The error measure for 
