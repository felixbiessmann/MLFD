% !Tex root = main.tex
\newpage
\section{Theory}
\emph{Functional Dependencies}(FDs) are a way of expressing ``a priori knowledge of restrictions or constraints on permissible sets of data''~\cite[p.~42]{MAI83} in relational database theory.
Since first work in the 1970s on schema normalization has been done, FDs have proven to be very useful for schema normalization.


\subsection{Relational Database Theory}
In order to give a definition of FDs, they need to be put in context to the domain they stem from: relational database theory. Some basic concepts will be introduced in this section.


\subsubsection{Relation Scheme}
A \emph{relation scheme}\footnote{also called \emph{relational schema} in literature\cite[p.21]{ABE19} } \(\boldsymbol{R}\) is a finite set of \emph{attribute names} \(\{A_1, A_2, \dots, A_n\}\), where to each attribute name \(A_i\) corresponds a set \(D_i\), called \emph{domain} of \(A_i\), \(1 \leq i \leq n\).
Let \(\boldsymbol{D} = D_1 \cup D_2 \cup \dots \cup D_n$, then a \emph{relation} \(r\) on relation scheme \(\boldsymbol{R}\) is a finite set of mappings \(\{t_1, t_2, \dots, t_p\}\) from \(\boldsymbol{R}\) to \(\boldsymbol{D}\):

\begin{align*}
  &t_i: \boldsymbol{R} \to \boldsymbol{D},
\end{align*}

where we call those mappings \emph{tuples} under the constraint that~\cite[p.2]{MAI83}

\begin{align*}
    t(A_i) \subseteq D_i.
\end{align*}

In application, attribute names are commonly called \emph{column name} or \emph{column attribute}.
One can think of them as labels of data that is stored in the respective column.


\subsubsection{Keys}
A \emph{key} on a relation \( r \) on a relation scheme \( R \) is a subset \( K = \{ B_1, B_2, \dots, B_m \} \) with the property that for any tuple \( t_i \in \{ t_1, t_2, \dots, t_3 \} \) the relation

\begin{align*}
    t_i(B_k) = t_j(B_k) \Rightarrow t_i \equiv t_j
\end{align*}

holds for any single \( B_k \in K \). In other words, any \( K \)-value of a tuple identifies that tuple uniquely.~\cite[p.~4]{MAI83} \\

Having defined both \emph{ralation scheme} and \emph{keys}, it is now possible to introduce the more complex conepts of \emph{relational databases} and \emph{functional dependencies}.


\subsubsection{Definition of a Relational Database}
When real-world data used by one or multiple application/s is stored on a machine according to the relational model, it is usually stored in a relational database.
According to the definition of a relation scheme \(R\), one can formally introduce databases and database schemes: \\

We assume that \(R\) is composed of two parts, \(S\) and \(\boldsymbol{K}\). We call \(S\) a \emph{set of attributes} and \(\boldsymbol{K}\) a \emph{set of designated keys} and describe this composition by writing \(R = (S, \boldsymbol{K})\).
A \emph{relational database scheme} \textbf{R} over \textbf{U} can now be defined as a collection of relation schemes \(\{R_1, R_1, \dots, R_p\}\), where \(R_i = (S_i, \boldsymbol{K}_i)\), \(1 \leq i, j \leq p\),

\begin{align*}
    \bigcup^{p}_{i=1} S_i = \boldsymbol{U}.
\end{align*}

We demand that \(S_i \neq S_j\) if \(i \neq j\). \\

A \emph{relational database} \( d \) on a \emph{database scheme} \textbf{R} is a collection of relations \( d~=~\{r_1, r_2, \dots, r_p \} \) such that for each relation scheme \(R = (S, \boldsymbol{K}) \) in \textbf{R} there is a relation \(r\) in \(d\) such that \(r\) is a relation on \(S\) that satisfies every \emph{key} in \(\boldsymbol{K}\).~\cite[p.~94]{MAI83}


\subsubsection{Definition of a Functional Dependency}
Consider a relation \(r\) on scheme \(\boldsymbol{R}\) with subset \(X \subseteq \boldsymbol{R}\) and a single attribute \(A_i \in \boldsymbol{R}\).
A FD \(X \to A\) is said to be \emph{valid} in \(r\), if and only if

\begin{align}
    t_i[X] = t_j[X] \Rightarrow t_i[A] = t_j[A] \label{eq:fd-condition}
\end{align}

holds for all all pairs of distinct tuples \(t_i,t_j \in r\).\cite[p.~21]{ABE19}
We say that \(X\) \emph{functionally determines} \(A\)\cite[p.~43]{MAI83} and name \(X\) the \emph{left side}, whilst calling \(A\) the \emph{right side}.\\

\begin{table}[ht]
    \centering
    \begin{tabular}{lcccc}
        \toprule
        & \multicolumn{3}{c}{left side} & \multicolumn{1}{c}{right side} \\ \cmidrule(lr{.75em}){2-3} \cmidrule(lr{.75em}){4-5}
        ID & FIRST NAME & LAST NAME & TOWN & ZIP \\
        \midrule
        1 & Alice & Smith & Munich & 19139 \\
        2 & Peter& Meyer & Muinch & 19139 \\
        3 & Ana & Parker & Munich & 19139  \\
        4 & John & Pick & Berlin & 12055 \\
        \bottomrule
    \end{tabular}
    \caption{Even though column ZIP functionally determines column Town (and vice-versa), a FD is not capable of displaying this fact - a typing error invalidates the FD.}
    \label{tab:example-afd-necessity}
\end{table}


\section{FDs in Application}
FDs are primarily used in database normalization,\cite[p.~1]{CAR16} but also find application in the field of data profiling, where ``any dependency can be turned into a rule to check for errors in the data''.\cite[p.~9]{ABE19}


\subsection{Normalization}
When introducing the relational database model in his 1970 article ``A relational model of data for large shared data banks'', Edgar F. Codd formalized database normalization alongside.\cite{COD70}
Describing what will be know to academia as \textbf{First normal form} (1NF), Codd states that ``problems treated [when normalizing databases] are those of \emph{data independence}'', aiming to protect future users of large databases ``from having to know how the data is organized in the machine''. \cite[p.~1]{COD70} \\

Being designed for as efficient as possible query handling, databases at the time were structured hierarchically or navigationally.
While this yielded good performance in times when computing time was very expensive, it came with a heavy cost of complexity:
``Teams of programmers were needed to express queries to extract meaningful information. [\dots] Such databases [\dots] were absolutely inflexible[y]''.\cite{IBM03}

Update-, insertion- and deletion anomalies can be prevented when normalizing a relational database. \cite[p.~75]{KLE11}

\subsubsection{First Normal Form}
A relation scheme $R$ is in \emph{First Normal Form} (1NF), if values in \(dom(A)\) are atomic for every attribute \(A\) in \(R\). \cite[p.~96]{MAI83}
Consider table \ref{tab:first-normal-form} which represents two relational database schemes.
It serves as an example of what is called \emph{atomic} and \emph{compound} data in the Relational Database model. \cite[p.~6]{COD90}

\begin{table}[ht]
    \centering
    \ra{1.3}
    \begin{tabular}{@{}rlllllll@{}}\toprule
    & \multicolumn{3}{c}{compound scheme} & \phantom{abc}& \multicolumn{2}{c}{atomic scheme} \\
    \cmidrule{2-3} \cmidrule{5-8}
    & NAME & ADRESS && PRENAME & SURNAME & TOWN & STREET   \\ \midrule
    1 & Alice Smith & Munich, Alicestr. && Alice & Smith & Alicestr. & Munich \\
2 &  Peter Meyer & Munich, Peterstr. && Peter & Meyer & Munich & Peterstr. \\
3 & Ana Parker & Munich, Anastr. && Ana & Parker & Munich & Anastr. \\
4 & John Pick & Berlin, Johnstr. && John & Pick & Berlin & Johnstr. \\
\bottomrule
\end{tabular}
\caption{The compound attributes ADRESS and NAME can be split into their atomic components TOWN and STREET as well as PRENAME and SURNAME, respectively.}
\label{tab:first-normal-form}
\end{table}
While the compound scheme's attributes can be decomposed into several other attributes, whereas an atomic attribute cannot be further split into any meaningful smaller components.\\

For a database it is said that the database in 1NF if every relation scheme in the database scheme is in 1NF.
1NF is the very foundation of the Relational Model, where the only type of compound data is the relation.\cite[p.~6]{COD90}

\subsubsection{Second Normal Form (2NF)}
A relation scheme \(R\) is said to be in \emph{Secon Normal Form} (2NF) in respect to a set of FDs \(F\), if it is in 1NF and every nonprime attribute is fully dependent on every key of of \(R\).\cite[p.~99]{MAI83}

\subsubsection{Third Normal Form (3NF)}

\subsection{Approximate Functional Dependencies}
In the field of data profiling an extensive body of theory and algorithms for FD detection has been created in the past decades.\cite{PAP15}
These mainly consider FDs as defined in formula \ref{eq:fd-condition}.
Howevever, the strict detection of FDs yields results that are solely applicable in a strictly controlled environment.
Real-world datasets faced by data-scientists or database engineers are often \emph{noisy}.
Entries might be corrupted by missing data, wrongly entered entries or incomplete datasets.
Inconsistencies are to be expected.
Thus, functionally dependent column-combinations might not be detected as such. This may result in misleading insights when searching for FDs. \\

To illustrate this, table \ref{tab:example-afd-necessity} shows an example of noisy data.
The potential FD \textbf{Town} \(\to\) \textbf{ZIP} is not captured by the definition given in equation \ref{eq:fd-condition}.
Due to a type-error, the potential FD is invalidated.
To still capture meta-information, a different dependency-measure than given in equation \ref{eq:fd-condition} is needed. \\

\emph{Approximate FDs} (AFDs), sometimes called \emph{Relaxed FDs}, improve the applicability of FDs, ``in that they relax one or more constraints of the canonical FDs''\cite[p.~1]{CAR16}. While there are AFDs introducing general error measures, others are defined ``aiming to solve specific problems''\cite[p.~1]{CAR16}. \\

\begin{table}[ht]
    \centering
    \begin{tabular}{lcccc}
        \toprule
        & \multicolumn{3}{c}{Data} \\ \cmidrule(lr){2-5}
        ID & First name & Last name & Town & ZIP \\
        \midrule
        1 & Alice & Smith & Munich & 19139 \\
        \textbf{2} & \textbf{Peter}& \textbf{Meyer} &
        \textbf{Muinch} & \textbf{19139} \\
        3 & Ana & Parker & Munich & 19139  \\
        4 & John & Pick & Berlin & 12055 \\
        \bottomrule
    \end{tabular}
    \caption{Even though column ZIP functionally determines column Town (and vice-versa), a FD is not capable of displaying this fact - a typing error invalidates the FD.}
    \label{tab:example-afd-necessity}
\end{table}

The error measure for this is not trivial at all. While F1-measures can be established for non-categorical cases, comparing results for different data-types tricky.

\subsection{FD Imputer}


\begin{algorithm}[H]
    \DontPrintSemicolon
    \SetAlgoLined
    \KwResult{Imputed column of a relational database}
    \KwData{Relational database}
    \BlankLine

    Split relational database in test-set and train-set\;
    Detect FDs in train-set\;
    \For{row in test-set}{
        Find row in train-set with equal LHS combination\;
        \If{matching LHS combination found}{
            impute with RHS from train-set\;
        }
        \If{No matching LHS combination found in train-set}{
            impute with NaN\;
        }
    }
    \caption{An imputer operating on Functional Dependencies}
\end{algorithm}
